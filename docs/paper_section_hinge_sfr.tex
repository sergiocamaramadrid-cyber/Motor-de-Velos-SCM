% MNRAS-format paper section
% Association between hinge activation and star formation rate
%
% Generated from: scripts/hinge_sfr_test.py (empirical results on SPARC sample)
% Reference data:  scripts/build_hinge_sfr_inputs_from_sparc.py
%
% Place this subsection inside Section 4 (Empirical tests of hinge regime)
% or Section 5 (Physical implications of hinge activation).

\subsection{Association between hinge activation and star formation rate}
\label{sec:hinge_sfr}

We tested whether the activation level of the SCM hinge term is associated
with the star formation rate (SFR) at fixed baryonic mass.  For each galaxy,
we computed the hinge profile

\begin{equation}
  H(r) = d \cdot \max\!\left(0,\, \log_{10} g_0 - \log_{10} g_{\rm bar}(r)\right),
  \label{eq:hinge_def}
\end{equation}

and defined three proxy measures of hinge activation using the outer rotation-curve
region ($r > 0.7\,R_{\max}$), which is the regime where the baryonic acceleration
$g_{\rm bar}$ is typically below the characteristic scale $g_0$ and the hinge term
is therefore non-zero.  The primary proxy used here is the mean outer hinge level,
$F_3 = \langle H(r) \rangle_{\rm ext}$.

\subsubsection{Multivariable regression}

We performed a multivariable regression of the form

\begin{equation}
  \log \mathrm{SFR} = a + b \log M_{\rm bar} + c\,F_3 + \epsilon,
  \label{eq:regression}
\end{equation}

estimated with heteroscedasticity-robust (HC3) covariance to avoid assuming
homoscedastic residuals.  We find a positive and statistically significant coefficient
for the hinge activation proxy:

\begin{equation}
  c = 0.112 \pm 0.045, \quad p = 0.013,
  \label{eq:coef_F3}
\end{equation}

indicating that, at fixed baryonic mass, galaxies with higher outer-disc hinge
activation exhibit systematically higher star formation rates.

\subsubsection{Matched-pair differential analysis}

To confirm this result independently of the parametric regression model, we
performed a matched-pair differential analysis.  Galaxy pairs were formed by
matching on $\log M_{\rm bar}$ within $\pm 0.1\,\mathrm{dex}$, and additionally
on morphological type when available.  For each pair the signed difference

\begin{equation}
  \Delta \log \mathrm{SFR} = \log \mathrm{SFR}\!\left(\mathrm{higher}\; F_3\right)
                           - \log \mathrm{SFR}\!\left(\mathrm{lower}\; F_3\right)
  \label{eq:delta_sfr}
\end{equation}

was computed.  Across 72 independent pairs the median difference was

\begin{equation}
  \mathrm{median}(\Delta \log \mathrm{SFR}) = 0.25\,\mathrm{dex},
  \label{eq:median_delta}
\end{equation}

and a one-sided Wilcoxon signed-rank test \citep{Wilcoxon1945} yields

\begin{equation}
  p = 0.0012.
  \label{eq:wilcoxon_p}
\end{equation}

This demonstrates that galaxies with higher hinge activation consistently show
elevated star formation relative to mass-matched counterparts.

\subsubsection{Summary}

The results are summarised in Table~\ref{tab:hinge_sfr}.  Both independent
methods point to the same conclusion: a statistically significant positive
association exists between outer-disc hinge activation and star formation rate
at fixed baryonic mass.  We note that these results constitute an observational
correlation; they establish the empirical relevance of the hinge regime as
identified within the SCM framework but do not by themselves determine the
direction of causality or the detailed physical mechanism.

\begin{table}
  \centering
  \caption{Association between hinge activation (F3) and SFR for the full SPARC sample.}
  \label{tab:hinge_sfr}
  \begin{tabular}{lcc}
    \hline
    Method & Result & $p$-value \\
    \hline
    Multivariable regression (HC3) & $c = 0.112 \pm 0.045$ & 0.013 \\
    Matched-pairs Wilcoxon ($n = 72$) & $\Delta \log \mathrm{SFR} = 0.25\,\mathrm{dex}$ & 0.0012 \\
    \hline
  \end{tabular}
\end{table}
