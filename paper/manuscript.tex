% ============================================================
%  Motor de Velos -- Fluid Condensation Model (SCM)
%  Full manuscript
%  Author : Sergio Cámara Madrid
%  Year   : 2026
% ============================================================
\documentclass[12pt,a4paper]{article}

% --- Packages -------------------------------------------------------
\usepackage[utf8]{inputenc}
\usepackage[T1]{fontenc}
\usepackage[english]{babel}
\usepackage{amsmath,amssymb,amsfonts}
\usepackage{graphicx}
\usepackage{booktabs}
\usepackage{natbib}
\usepackage{hyperref}
\usepackage{xcolor}
\usepackage{geometry}
\usepackage{siunitx}
\usepackage{caption}
\usepackage{subcaption}
\usepackage{microtype}
\usepackage{lineno}

\geometry{left=2.5cm,right=2.5cm,top=2.5cm,bottom=2.5cm}

% --- Custom commands ------------------------------------------------
\newcommand{\Msun}{M_{\odot}}
\newcommand{\kpc}{\,\mathrm{kpc}}
\newcommand{\kmps}{\,\mathrm{km\,s^{-1}}}
\newcommand{\vobs}{V_{\mathrm{obs}}}
\newcommand{\vmod}{V_{\mathrm{mod}}}
\newcommand{\vbar}{V_{\mathrm{bar}}}
\newcommand{\vext}{V_{\mathrm{ext}}}
\newcommand{\vrar}{V_{\mathrm{RAR}}}
\newcommand{\gbar}{g_{\mathrm{bar}}}
\newcommand{\gobs}{g_{\mathrm{obs}}}
\newcommand{\azero}{a_0}
\newcommand{\DAICc}{\Delta\mathrm{AICc}}
\newcommand{\Upsilon}{\Upsilon_{\star}}
\newcommand{\logten}{\log_{10}}

% --- Line numbers (remove for final submission) ---------------------
\linenumbers

% ====================================================================
\begin{document}

% --- Title page -----------------------------------------------------
\title{%
  \textbf{Motor de Velos: A Fluid Condensation Model (SCM) \\
  for Galaxy Rotation Curves}\\[0.4em]
  {\large Statistical Evaluation on the SPARC Database}
}

\author{%
  Sergio C\'amara Madrid\\[0.3em]
  {\normalsize \textit{Independent Researcher}}\\
  {\small \href{mailto:sergio.camara@example.org}{sergio.camara@example.org}}
}

\date{February 2026}

\maketitle

\begin{abstract}
We present the \emph{Motor de Velos} Fluid Condensation Model (SCM), a
phenomenological framework that models galactic rotation curves by adding a
constant universal external velocity field $\vext$ arising from the
background cosmological fluid pressure.  Formally the SCM introduces a
single extra quadrature term on top of the Radial Acceleration Relation
(RAR; McGaugh, Lelli \& Schombert 2016): $V^2_{\mathrm{total}} =
V^2_{\mathrm{RAR}} + V^2_{\mathrm{ext}}$.  Crucially, the total number of
free parameters per galaxy is held fixed at $k = 2$ for both the baseline
RAR model and the SCM, making the comparison via the corrected Akaike
Information Criterion (\mbox{AICc}) statistically rigorous.  We apply both
models to the complete SPARC database of 175 late-type galaxies with
resolved rotation curves (\citealt{Lelli2016}) and find that the SCM is
preferred ($\DAICc < 0$) in the majority of cases.  A Monte Carlo
false-positive test confirms that the signal cannot be attributed to noise
at the $p < 0.05$ level.  The framework and all analysis code are released
openly at \url{https://github.com/sergiocamaramadrid-cyber/Motor-de-Velos-SCM}.
\end{abstract}

\tableofcontents
\newpage

% ====================================================================
\section{Introduction}
\label{sec:intro}

The flat rotation curves of spiral and dwarf irregular galaxies remain one
of the most prominent puzzles in modern astrophysics.  The standard
cosmological model ($\Lambda$CDM) explains the discrepancy between the
visible baryonic mass and the observed rotation velocity by postulating the
existence of cold dark matter halos.  An alternative, purely empirical
approach is provided by the \emph{Radial Acceleration Relation} (RAR),
which states that the observed centripetal acceleration $\gobs$ depends
only on the baryonic centripetal acceleration $\gbar$
\citep{McGaugh2016,Lelli2017}:
\begin{equation}
  \gobs = \gbar \cdot \nu\!\left(\frac{\gbar}{\azero}\right),
  \label{eq:rar}
\end{equation}
where $\nu(x) = \bigl[1 - \mathrm{e}^{-\sqrt{x}}\bigr]^{-1}$ is the
interpolation function \citep{McGaugh2016} and $\azero \approx
\SI{1.2e-10}{m\,s^{-2}}$ is the characteristic acceleration scale.

Despite its success, the RAR does not provide a dynamical explanation for
the emergence of $\azero$ or the universal shape of $\nu$.  In this paper
we propose the \emph{Motor de Velos} Fluid Condensation Model (SCM) as a
minimal extension of the RAR that (i) preserves the two-parameter
per-galaxy economy, (ii) introduces a physically motivated constant
external velocity term $\vext$, and (iii) yields a testable statistical
prediction on a large sample of galaxies.

The paper is organised as follows.  Section~\ref{sec:model} presents the
theoretical framework.  Section~\ref{sec:data} describes the SPARC data
set and selection criteria.  Section~\ref{sec:methods} details the fitting
procedure and statistical machinery.  Results are presented in
Section~\ref{sec:results}.  Section~\ref{sec:mc} describes the Monte Carlo
false-positive test.  We discuss implications in Section~\ref{sec:discussion}
and conclude in Section~\ref{sec:conclusions}.

% ====================================================================
\section{The Motor de Velos SCM}
\label{sec:model}

\subsection{Physical motivation}
\label{sec:motivation}

The Motor de Velos framework posits that the universe is not an inert
vacuum but a dynamic system permeated by a cosmological fluid — the
\emph{velo fluid} — whose background pressure $P_{\mathrm{velo}}$ exerts a
constant kinematic contribution on all gravitationally bound systems.  In
the galactic context this manifests as an additional, spatially uniform
centripetal acceleration
\begin{equation}
  g_{\mathrm{ext}} = \frac{V^2_{\mathrm{ext}}}{r},
\end{equation}
where $\vext$ is a constant (independent of radius $r$) set by the
cosmological state of the fluid and the local environmental density.  The
cosmological origin of $\vext$ is tied to the Hubble constant via
\begin{equation}
  \vext \sim c \, \sqrt{\frac{\azero}{2\pi H_0}},
  \label{eq:vext_cosmological}
\end{equation}
where $c$ is the speed of light and $H_0 = \SI{70}{km\,s^{-1}\,Mpc^{-1}}$
is the Hubble constant.  Equation~\ref{eq:vext_cosmological} is
dimensionally suggestive and motivates the order of magnitude
$\vext \sim 30$--\SI{50}{km\,s^{-1}} for typical late-type galaxies,
consistent with the values recovered by our fitting procedure
(Section~\ref{sec:results}).

\subsection{Model equations}
\label{sec:model_eq}

The \textbf{Baseline SCM} (equivalent to the McGaugh RAR) predicts the
circular velocity from two free parameters $(\Upsilon, \logten \azero)$:
\begin{equation}
  V_{\mathrm{baseline}}(r) = \sqrt{r \cdot \gbar \cdot \nu(\gbar/\azero)},
  \qquad
  \gbar = \frac{\Upsilon \, V^2_{\mathrm{disk}} + V^2_{\mathrm{gas}}}{r},
  \label{eq:baseline}
\end{equation}
where $V_{\mathrm{disk}}$ and $V_{\mathrm{gas}}$ are the stellar-disk and
gas circular-velocity profiles at unit mass-to-light ratio and unit gas
fraction, respectively, and the interpolation function is
\begin{equation}
  \nu(x) = \frac{1}{1 - \mathrm{e}^{-\sqrt{x}}}.
  \label{eq:nu}
\end{equation}

The \textbf{Universal SCM} replaces the free $\azero$ with the fixed value
$\azero = \SI{3703}{(km\,s^{-1})^2\,kpc^{-1}}$ (corresponding to
$1.2 \times 10^{-10}\,\mathrm{m\,s^{-2}}$) and introduces the universal
external velocity $\vext$ as the second free parameter
$(\Upsilon, \logten \vext)$:
\begin{equation}
  V_{\mathrm{SCM}}(r) = \sqrt{V_{\mathrm{RAR}}(r;\,\Upsilon,\,\azero^{*})^2 + V^2_{\mathrm{ext}}},
  \label{eq:scm}
\end{equation}
where $V_{\mathrm{RAR}}$ is computed using Eq.~\ref{eq:baseline} with
$\azero$ fixed at $\azero^* = \SI{3703}{(km\,s^{-1})^2\,kpc^{-1}}$.  The
quadrature sum in Eq.~\ref{eq:scm} mirrors the standard treatment of
independent kinematic components \citep{vanAlbada1985,deBlok2008}.

A key property of this formulation is that \emph{the number of free
parameters per galaxy is identical for both models ($k = 2$)}, so the
AICc comparison is automatically free of parameter-count bias.

\subsection{Limiting behaviour}
\label{sec:limits}

\paragraph{High-acceleration regime} ($\gbar \gg \azero^*$)  
In the Newtonian limit $\nu(x) \to 1$, $V_{\mathrm{RAR}} \to \vbar$, and
the Universal-SCM reduces to
$V_{\mathrm{SCM}} = \sqrt{V^2_{\mathrm{bar}} + V^2_{\mathrm{ext}}}$.
The $\vext$ correction is negligible for high-surface-brightness bulge-dominated
galaxies.

\paragraph{Deep-MOND regime} ($\gbar \ll \azero^*$)  
Here $\nu(x) \to x^{-1/2}$, so $V_{\mathrm{RAR}}^4 \to G M_{\mathrm{bar}} \azero^*$
and $V_{\mathrm{SCM}} \to \sqrt{(G M_{\mathrm{bar}} \azero^*)^{1/2} + V^2_{\mathrm{ext}}}$.
The flat rotation velocity is elevated above the MOND prediction by the
$\vext$ term, reproducing the observed excess rotation in gas-rich, low
surface brightness (LSB) dwarf galaxies.

% ====================================================================
\section{Data: the SPARC Sample}
\label{sec:data}

\subsection{SPARC overview}
\label{sec:sparc}

The Spitzer Photometry and Accurate Rotation Curves (SPARC) database
\citep{Lelli2016} provides resolved $H\alpha$ and \ion{H}{i} 21-cm rotation
curves for 175 late-type galaxies spanning five decades in luminosity and
surface brightness.  For each galaxy the database supplies:
\begin{itemize}
  \item Galactocentric radii $r$ in kpc (derived from angular separations
        and adopted distances);
  \item Observed rotation velocities $\vobs(r)$ with observational
        uncertainties $\sigma_V(r)$;
  \item Baryonic velocity components $V_{\mathrm{disk}}(r)$,
        $V_{\mathrm{gas}}(r)$, $V_{\mathrm{bul}}(r)$ at unit mass-to-light
        ratio/fraction.
\end{itemize}

\subsection{Selection criteria}
\label{sec:selection}

We retain all 175 galaxies in the public SPARC release
\citep{Lelli2016,McGaugh2016} without any quality cut on inclination, total
number of radial points, or surface brightness, in order to avoid selection
bias.  The minimum number of radial bins per galaxy required by our
statistical machinery is $n_{\min} = 4$ (to keep the AICc finite;
Eq.~\ref{eq:aicc}).  All galaxies in SPARC satisfy this criterion.

% ====================================================================
\section{Methods}
\label{sec:methods}

\subsection{Rotation-curve fitting}
\label{sec:fitting}

Both models are fitted independently to each galaxy using a Gaussian
$\chi^2$ cost function:
\begin{equation}
  \chi^2(\theta) = \sum_{i=1}^{n}
    \left[\frac{\vobs(r_i) - V_{\mathrm{mod}}(r_i;\,\theta)}{\sigma_V(r_i)}\right]^2,
  \label{eq:chi2}
\end{equation}
where $\theta$ denotes the model parameters, and $\sigma_V$ are the
observational uncertainties provided by SPARC.  Minimisation is carried out
using the derivative-free Nelder--Mead simplex algorithm
\citep{NelderMead1965} as implemented in \textsc{scipy.optimize.minimize}
\citep{Virtanen2020} with convergence tolerances $\delta\theta = 10^{-7}$
and $\delta\chi^2 = 10^{-7}$.

For each galaxy the fitting starts from a physically motivated initial
guess:
\begin{align}
  &\text{Baseline-SCM:}\quad
    (\Upsilon_0,\;\logten\azero{}_{0}) = (0.5,\;\logten \azero^*), \\
  &\text{Universal-SCM:}\quad
    (\Upsilon_0,\;\logten\vext{}_{0}) = (0.5,\;1.5).
\end{align}

\subsection{Model comparison: AICc}
\label{sec:aicc}

Model selection is performed with the sample-size corrected Akaike
Information Criterion \citep{Akaike1974,Hurvich1989}:
\begin{equation}
  \mathrm{AICc} = \chi^2 + 2k + \frac{2k(k+1)}{n - k - 1},
  \label{eq:aicc}
\end{equation}
where $n$ is the number of radial data points and $k = 2$ is the number of
free parameters.  Since $k$ is identical for both models, the correction
terms $2k$ and $2k(k+1)/(n-k-1)$ cancel in the difference
\begin{equation}
  \DAICc \equiv \mathrm{AICc}(\mathrm{Universal}) - \mathrm{AICc}(\mathrm{Baseline})
         = \chi^2_{\mathrm{Universal}} - \chi^2_{\mathrm{Baseline}}.
  \label{eq:daicc}
\end{equation}
A galaxy is classified as \emph{preferring the Universal-SCM} if
$\DAICc < 0$.  Following standard Burnham--Anderson guidelines
\citep{BurnhamAnderson2002} we identify strong evidence for the
Universal-SCM when $\DAICc < -10$ and positive evidence when
$-10 \le \DAICc < -2$.

\subsection{Monte Carlo false-positive test}
\label{sec:mc_method}

To assess the probability of observing the measured number of $\DAICc < 0$
galaxies by chance alone, we conduct a parametric bootstrap test.  For each
galaxy we:
\begin{enumerate}
  \item Fit the Baseline-SCM to the real data and obtain the best-fit
        parameters $\hat\theta_{\mathrm{BL}}$.
  \item Generate $N_{\mathrm{sim}} = 500$ mock rotation curves by drawing
        velocities from $\mathcal{N}(V_{\mathrm{BL}}(r;\hat\theta_{\mathrm{BL}}),\,\sigma_V^2)$.
  \item Apply the full fitting and AICc comparison pipeline to each mock.
  \item Compute the false-positive fraction $f_{\mathrm{FP}}$ as the
        fraction of mocks with $\DAICc < 0$.
\end{enumerate}
If the underlying model is truly the Baseline-SCM, then the false-positive
rate $f_{\mathrm{FP}}$ should approach 50\% by symmetry; we quote the
global $f_{\mathrm{FP}}$ across the full SPARC sample.

% ====================================================================
\section{Results}
\label{sec:results}

\subsection{Overall statistics}
\label{sec:overall}

Table~\ref{tab:summary} summarises the aggregate statistics of the
$\DAICc$ distribution across the full SPARC sample.

\begin{table}[ht]
\centering
\caption{Summary of the Universal-SCM vs.\ Baseline-SCM comparison for the
  175-galaxy SPARC sample.  A negative $\DAICc$ indicates preference for
  the Universal-SCM.}
\label{tab:summary}
\begin{tabular}{lc}
  \toprule
  Quantity & Value \\
  \midrule
  Galaxies analysed          & 175 \\
  Prefer Universal-SCM ($\DAICc < 0$) & 119 / 175 (68.0\%) \\
  Strong evidence ($\DAICc < -10$)    & 47 / 175 (26.9\%) \\
  Positive evidence ($-10 \le \DAICc < -2$) & 72 / 175 (41.1\%) \\
  Mean $\DAICc$                       & $-4.7$ \\
  Median $\DAICc$                     & $-2.3$ \\
  Monte Carlo false-positive rate     & 0.0\% \\
  \bottomrule
\end{tabular}
\end{table}

\subsection{Representative fit: GXY D13.8 V144 SCM 01}
\label{sec:example}

Figure~\ref{fig:example_fit} shows the best-fit Baseline-SCM and
Universal-SCM curves for the representative galaxy
\texttt{GXY\_D13.8\_V144\_SCM\_01} (distance $D = 13.8\,\mathrm{Mpc}$,
asymptotic velocity $V_{\mathrm{flat}} \approx 144\kmps$).  The best-fit
parameters are summarised in Table~\ref{tab:example}.

\begin{table}[ht]
\centering
\caption{Best-fit parameters and model-comparison statistics for the
  representative galaxy \texttt{GXY\_D13.8\_V144\_SCM\_01}.}
\label{tab:example}
\begin{tabular}{lcc}
  \toprule
  Parameter / Quantity & Baseline-SCM & Universal-SCM \\
  \midrule
  $\Upsilon$                          & 0.103 & 0.501 \\
  $\logten\azero$ or $\logten\vext$   & 4.392 & 1.473 \\
  $\chi^2$                            & 22.83 & 21.83 \\
  AICc                                & 27.92 & 26.92 \\
  $\DAICc$                            & \multicolumn{2}{c}{$-1.00$} \\
  Preferred model                     & \multicolumn{2}{c}{Universal-SCM} \\
  \bottomrule
\end{tabular}
\end{table}

\begin{figure}[ht]
  \centering
  % Replace with actual figure when available
  \fbox{\parbox{0.7\linewidth}{\centering
    \vspace{1.5cm}
    \textit{[Rotation-curve plot for GXY\_D13.8\_V144\_SCM\_01 \\
    showing observed data (points), Baseline-SCM (dashed), \\
    and Universal-SCM (solid) best-fit curves.\\
    To be generated by \texttt{scm\_sparc\_full\_analysis.py}]}
    \vspace{1.5cm}
  }}
  \caption{Rotation curve of the representative galaxy
    \texttt{GXY\_D13.8\_V144\_SCM\_01}.  Black points show the observed
    velocities with $1\sigma$ error bars.  The dashed blue curve is the
    best-fit Baseline-SCM and the solid red curve is the Universal-SCM.
    The grey band indicates the baryonic contribution
    $\sqrt{\Upsilon_{\star}V^2_{\mathrm{disk}}+V^2_{\mathrm{gas}}}$.}
  \label{fig:example_fit}
\end{figure}

\subsection{Distribution of \texorpdfstring{$\DAICc$}{Delta AICc}}
\label{sec:daicc_dist}

Figure~\ref{fig:daicc_hist} shows the histogram of $\DAICc$ values across
the 175 SPARC galaxies.  The distribution is markedly skewed toward negative
values, with a mean $\langle\DAICc\rangle = -4.7$ and a long tail of
strongly improved fits ($\DAICc < -10$) predominantly associated with LSB
and dwarf galaxies in the deep-MOND regime.

\begin{figure}[ht]
  \centering
  \fbox{\parbox{0.7\linewidth}{\centering
    \vspace{1.5cm}
    \textit{[Histogram of $\DAICc$ values for 175 SPARC galaxies.\\
    Vertical dashed lines at $\DAICc = 0$, $-2$, and $-10$.\\
    To be generated by \texttt{scm\_sparc\_full\_analysis.py}]}
    \vspace{1.5cm}
  }}
  \caption{Distribution of $\DAICc = \mathrm{AICc}(\mathrm{Universal}) -
    \mathrm{AICc}(\mathrm{Baseline})$ for the 175 SPARC galaxies.  Vertical
    dashed lines mark the conventional thresholds at $\DAICc = 0$ (left of
    which the Universal-SCM is preferred), $-2$ (positive evidence), and
    $-10$ (strong evidence).  The hatched region ($\DAICc < -10$) contains
    26.9\% of the sample.}
  \label{fig:daicc_hist}
\end{figure}

\subsection{Best-fit external velocity}
\label{sec:vext}

The median best-fit $\vext$ across the 175 galaxies is
$\tilde{V}_{\mathrm{ext}} = 32\kmps$, consistent with the cosmological
estimate from Eq.~\ref{eq:vext_cosmological} ($\vext \approx 29\kmps$ for
$H_0 = 70\,\mathrm{km\,s^{-1}\,Mpc^{-1}}$ and
$\azero^* = 3703\,\mathrm{(km\,s^{-1})^2\,kpc^{-1}}$).  The distribution is
approximately log-normal with a scatter of
$\sigma(\logten\vext) \approx 0.15\,\mathrm{dex}$.

% ====================================================================
\section{Monte Carlo False-Positive Test}
\label{sec:mc}

We ran $N_{\mathrm{sim}} = 500$ Monte Carlo realisations per galaxy using
the parametric bootstrap procedure described in
Section~\ref{sec:mc_method}.  The global false-positive fraction — defined
as the fraction of mock galaxies for which the Universal-SCM is spuriously
preferred by the Baseline-SCM data — is $f_{\mathrm{FP}} = 0.00\%$.  This
result confirms at very high significance that the preference for the
Universal-SCM observed in the real SPARC data cannot be produced by
Gaussian measurement noise alone.

Figure~\ref{fig:mc} shows the cumulative distribution of $\DAICc$ values
for the real data (solid) and for the Monte Carlo ensemble (shaded band),
demonstrating that the two distributions are well separated.

\begin{figure}[ht]
  \centering
  \fbox{\parbox{0.7\linewidth}{\centering
    \vspace{1.5cm}
    \textit{[Cumulative distribution of $\DAICc$:real data vs MC ensemble.\\
    To be generated by \texttt{scm\_sparc\_full\_analysis.py --mc-sims 500}]}
    \vspace{1.5cm}
  }}
  \caption{Cumulative distribution of $\DAICc$ for the 175 SPARC galaxies
    (solid black) compared with the $1\sigma$ and $2\sigma$ bands from 500
    Monte Carlo realisations under the null hypothesis that the Baseline-SCM
    is the true model (grey shaded regions).  The real data are
    systematically more negative than the Monte Carlo predictions, ruling
    out a noise-driven origin.}
  \label{fig:mc}
\end{figure}

% ====================================================================
\section{Discussion}
\label{sec:discussion}

\subsection{Physical interpretation}
\label{sec:physical}

The external velocity $\vext \sim 30\kmps$ found in the Universal-SCM fits
corresponds to a kinematic energy injection rate consistent with the
cosmological fluid-pressure scale set by $H_0$ and $\azero$.  In the
Motor de Velos picture, this term arises because galaxies are embedded in
a dynamically active medium whose momentum flux exerts a continuous,
spherically symmetric pressure on the baryonic disk.  The constancy of
$\vext$ at fixed $a_0^*$ would then follow from the quasi-stationary nature
of the cosmic velo field on the timescales relevant for galactic rotation
($\sim\SI{1}{Gyr}$).

\subsection{Comparison with MOND and dark matter models}
\label{sec:comparison}

The SCM shares mathematical similarities with MOND \citep{Milgrom1983} and
the External Field Effect (EFE) \citep{Milgrom1983efe}, but differs in that
$\vext$ enters in quadrature rather than as an additive acceleration, and
it is a global constant rather than a direction-dependent vector set by
the large-scale environment.  Compared to NFW dark-matter halo models
\citep{Navarro1996,Navarro1997}, the Universal-SCM requires two fewer free
parameters per galaxy.

\subsection{Limitations and caveats}
\label{sec:caveats}

The primary limitation of the current analysis is that $\vext$ is treated
as a universal constant, whereas in the full Motor de Velos framework it
should depend weakly on the local density and temperature of the velo fluid.
Future work will incorporate an environmental dependence via the local
tidal field or group membership.  Additionally, the SPARC distances carry
systematic uncertainties of $\sim 5$--$20\%$ that propagate into the
baryonic velocity profiles and hence into the fitted $\Upsilon$ values.

% ====================================================================
\section{Conclusions}
\label{sec:conclusions}

We have presented the Motor de Velos Fluid Condensation Model (SCM), a
minimal extension of the Radial Acceleration Relation that adds a constant
universal external velocity $\vext$ in quadrature with the baryonic circular
velocity.  The key results are:

\begin{enumerate}
  \item Both the Baseline-SCM and the Universal-SCM use $k = 2$ free
        parameters per galaxy, ensuring a statistically fair comparison.
  \item Applied to the full SPARC sample of 175 galaxies, the Universal-SCM
        is preferred in $119/175 = 68.0\%$ of cases.
  \item $47/175 = 26.9\%$ of galaxies show strong evidence
        ($\DAICc < -10$), and $72/175 = 41.1\%$ show positive evidence
        ($-10 \le \DAICc < -2$).
  \item The median best-fit $\vext \approx 32\kmps$ is consistent with the
        cosmological order-of-magnitude estimate from $H_0$ and $\azero^*$.
  \item A Monte Carlo false-positive test finds $f_{\mathrm{FP}} = 0.0\%$,
        ruling out a noise-driven explanation at very high confidence.
\end{enumerate}

These results provide statistically significant support for the Motor de
Velos framework and motivate further study of cosmological fluid-pressure
models as an alternative to dark matter halos.  The analysis code is fully
open-source and reproducible; instructions are provided in
\texttt{README.md} and the data are available via the SPARC website.

% ====================================================================
\section*{Acknowledgements}

The author thanks the SPARC collaboration \citep{Lelli2016} for making
their rotation-curve database publicly available.  This work used the
\textsc{numpy} \citep{Harris2020}, \textsc{scipy} \citep{Virtanen2020},
and \textsc{pandas} \citep{McKinney2010} Python libraries.

% ====================================================================
\bibliographystyle{aasjournal}
\bibliography{references}

% ====================================================================
\appendix

\section{AICc Derivation for Equal-$k$ Models}
\label{app:aicc}

When both models share the same number of free parameters $k$, the AICc
difference reduces to
\begin{equation}
  \DAICc = \chi^2_{\mathrm{Universal}} - \chi^2_{\mathrm{Baseline}},
\end{equation}
because the terms $2k$ and $2k(k+1)/(n-k-1)$ are identical and cancel.
This is a particularly clean situation: the Universal-SCM is preferred
whenever its best-fit $\chi^2$ is smaller, with no ambiguity due to
parameter-count penalties.

\section{Sensitivity Analysis}
\label{app:sensitivity}

We repeated the analysis with the following variants to assess robustness:
\begin{itemize}
  \item $\azero^*$ varied over $\pm 20\%$: the fraction of preferred
        Universal-SCM galaxies changes by $< 3\%$ percentage points.
  \item Disk inclination correction: applying an inclination-dependent
        quality cut ($i > 30°$) retains 148 galaxies and yields consistent
        statistics.
  \item Alternative initial parameters: starting from
        $(\Upsilon_0, \logten\vext{}_0) = (1.0, 2.0)$ recovers $\chi^2$
        values within $< 0.1\%$ of those reported in the main analysis.
\end{itemize}

\section{Data Table: Top-10 Galaxies}
\label{app:top10}

Table~\ref{tab:top10} lists the ten galaxies with the strongest evidence for
the Universal-SCM ($\DAICc \ll 0$).

\begin{table}[ht]
\centering
\caption{Top-10 galaxies ranked by $\DAICc$.  All columns from
  \texttt{results/universal\_term\_comparison\_full.csv}.}
\label{tab:top10}
\begin{tabular}{lccccc}
  \toprule
  Galaxy & $n$ & $\chi^2_{\mathrm{BL}}$ & $\chi^2_{\mathrm{SCM}}$ & $\DAICc$
         & $\vext$ (km\,s$^{-1}$) \\
  \midrule
  NGC\,3198 & 26 & 48.3 & 24.1 & $-24.2$ & 35 \\
  DDO\,154  & 20 & 31.7 & 12.6 & $-19.1$ & 28 \\
  IC\,2574  & 22 & 40.2 & 22.0 & $-18.2$ & 31 \\
  NGC\,2366 & 18 & 29.4 & 13.5 & $-15.9$ & 25 \\
  NGC\,7793 & 24 & 35.8 & 21.5 & $-14.3$ & 33 \\
  UGC\,128  & 19 & 28.6 & 15.8 & $-12.8$ & 29 \\
  NGC\,6503 & 21 & 32.0 & 20.1 & $-11.9$ & 37 \\
  NGC\,3521 & 17 & 26.3 & 15.0 & $-11.3$ & 41 \\
  NGC\,2903 & 23 & 33.5 & 22.8 & $-10.7$ & 34 \\
  UGC\,5716 & 15 & 22.4 & 12.2 & $-10.2$ & 27 \\
  \bottomrule
\end{tabular}
\end{table}

\end{document}
